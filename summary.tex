\documentclass[12pt,a4paper]{article}
\usepackage[a4paper,textwidth=170mm,textheight=245mm]{geometry}
\usepackage[utf8]{inputenc}
\usepackage{fancyhdr}
\usepackage{datetime2}
\usepackage{parskip}
\usepackage{amsmath}


\pagestyle{fancy}
\fancyhf{}

\lhead{TANA09 - Datatekniska beräkningar}
\rhead{\today}
\setlength{\headheight}{15pt}

\cfoot{\thepage}

\begin{document}

\begin{center}
	\Huge
	\textbf{Sammanfattning}
	\\
	\large
	Michael Sörsäter
\end{center}

\section{Fel}
Typer av fel:
\begin{itemize}
\item{$R_X$ fel i resultatet, som härrör från fel i indata}
\item{$R_{XF}$ fel i resultatet, som härrör från fel i de använda funktionsvärdena}
\item{$R_B$ avrundningsfel}
\item{$R_T$ trunkeringsfel}
\end{itemize}

Närmevärde till x: $\bar{x}$

Absolut fel: $\Delta x = \bar{x}-x$

Relativt fel: $ \frac{\Delta x}{x}$

\subsection{Metodoberoende feluppskattning}
\large
$|x^{*}-\bar{x}| \leq \frac{f(\bar{x})}{f'(\bar{x})} $

\subsection{Feluppskattning för x i Ax = b}
\large
$\frac{||\Delta x||_\infty}{||x||_\infty} \leq ||A||_\infty ||A^{-1}||_\infty \frac{||\Delta b||_\infty}{||b||_\infty}$

\normalsize

\section{Interpolation}
\subsection{Linjär interpolation}
Beräkna polynomet:

\Large
$p_n(x) = c_0 + c_1(x-x_1) + c_2 (x-x_1)(x-x_2) + ...$

\normalsize
Beräkna konstanterna C genom att först sätta in $x_1$ (gör alla delar till 0 förutom $c_0$) och sedan vidare likadant.

Felet är den ``Extra term'' som inte ingår i splinen. $R_T$
För en linjär interpolation är det termen vid $c_2$

Vid en fullständig feluppskattning ska tre delar tas med.
Alla delar är såklart absolutbelopp.

\begin{itemize}
\item{$R_B$ Avrundningsfel.
Efter beräkningen P(a), har svaret t korrekta decimaler. \\Ger felet: $ R_B = 0.5 * 10^{-t}$}
\item{$R_T$ Trunkeringsfel.
(är ofta den dominerande delen)
Använd termen $c_n$ för att beräkna felet.
Ex: $ R_T \leq |c_2 (x-x_1)(x-x_2)|$
}
\item{$R_{XF}$ Fel i indatan.
Likadant som för $R_B$ antalet t decimaler som lägst givna i indatan.
$R_{XF} = 0.5 * 10^{-t}$}
\end{itemize}

Svaret blir sedan $R_{TOT} = R_B + R_T + R_{XF}$

\subsection{Fel i funktionsvärdet}

\Large
$A = 0.78 \pm 0.02$

\normalsize
Vad blir felet i $f(A)$

Använd maximalsfelsuppskattning:

\Large
$|\Delta f| \leq |\frac{\delta f}{\delta A}| |\Delta A| \approx |\frac{\delta P_n}{\delta A}| |\Delta x| = |c_1||\Delta A|$ 

\normalsize
Alltså, derivera med avseende på x. Kanske bara fungerar i fallet $P_1$

\section{LU-faktorisering}
Från matrisen A, beräkna $PA=LU$ där P är en permutationsmatris (ändrar om raderna).

$P_{12} = 
\begin{pmatrix}
0 & 1 & 0 \\
1 & 0 & 0 \\
0 & 0 & 1
\end{pmatrix}
,
L = 
\begin{pmatrix}
1 & 0 & 0 \\
a & 1 & 0 \\
b & c & 1
\end{pmatrix}
,
U = 
\begin{pmatrix}
a & b & c \\
0 & d & e \\
0 & 0 & f
\end{pmatrix}
$

Används för att kunna lösa Ax=b.

\Large
$Ax = b \Leftrightarrow PAx = Pb \Leftrightarrow LUx = Pb$

Sätt $Ux = y$, lös $Ly = Pb$ och sedan $Ux = y$

Använd maximalsfelsuppskattning:

\Large
$|\Delta f| \leq |\frac{\delta f}{\delta a}| |\Delta a| 
|\frac{\delta f}{\delta b}| |\Delta b|$ 

\section{Derivering}

\Large
$f = \frac{u}{v} $

$\frac{d}{dx}\frac{u}{v} = \frac{v\frac{du}{dx}-u\frac{dv}{dx}}{v^2}$

%	\frac{}{}
\section{Summa - restterm}
\normalsize
Summa som är konvergent

\Large
$$ S = \sum_{n=1}^{\infty} a_n $$
$$ S_N = \sum_{n=1}^{N} a_n$$
$$ R_N = S - S_N = \sum_{n=N+1}^{\infty} a_n $$


\normalsize
Hur uppskattas $R_N$ utan att räkna ut den?

\subsection{Alternerande}
Är serien konvergent och alternerande:

\Large
$$ |R_N| \leq |a_{N+1}|$$

\subsection{Postiv monotont avtagande}
$$ R_N = \sum_{N+1}^{\infty} f(n)X \leq \int_{N}^{\infty} f(x) dx $$

\normalsize
\section{Maclaurin}
\Large
$ e^x = \sum_{n=0}^{\infty} \frac{x^n}{n!} = 1 + x + \frac{x^2}{2!} + \frac{x^3}{3!} + \ldots $

$ sin(x) = \sum_{n=0}^{\infty} \frac{(-1)^n}{(2n+1)!} x^{2n+1} = x - \frac{x^3}{3!} + \frac{x^5}{5!} - \frac{x^7}{7!} + \ldots$

$ cosx) = \sum_{n=0}^{\infty} \frac{(-1)^n}{(2n)!} x^{2n} = 1 - \frac{x^2}{2!} + \frac{x^4}{4!} - \frac{x^6}{6!} + \ldots$

\end{document}
